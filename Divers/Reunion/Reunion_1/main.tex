% --------------------------------------------------------------
% This is all preamble stuff that you don't have to worry about.
% Head down to where it says "Start here"
% --------------------------------------------------------------
 
\documentclass[12pt]{article}
 
\usepackage[margin=1in]{geometry} 
\usepackage{amsmath,amsthm,amssymb}
 
\newcommand{\N}{\mathbb{N}}
\newcommand{\Z}{\mathbb{Z}}
 
\newenvironment{theorem}[2][Theorem]{\begin{trivlist}
\item[\hskip \labelsep {\bfseries #1}\hskip \labelsep {\bfseries #2.}]}{\end{trivlist}}
\newenvironment{lemma}[2][Lemma]{\begin{trivlist}
\item[\hskip \labelsep {\bfseries #1}\hskip \labelsep {\bfseries #2.}]}{\end{trivlist}}
\newenvironment{exercise}[2][Exercise]{\begin{trivlist}
\item[\hskip \labelsep {\bfseries #1}\hskip \labelsep {\bfseries #2.}]}{\end{trivlist}}
\newenvironment{problem}[2][Problem]{\begin{trivlist}
\item[\hskip \labelsep {\bfseries #1}\hskip \labelsep {\bfseries #2.}]}{\end{trivlist}}
\newenvironment{question}[2][Question]{\begin{trivlist}
\item[\hskip \labelsep {\bfseries #1}\hskip \labelsep {\bfseries #2.}]}{\end{trivlist}}
\newenvironment{corollary}[2][Corollary]{\begin{trivlist}
\item[\hskip \labelsep {\bfseries #1}\hskip \labelsep {\bfseries #2.}]}{\end{trivlist}}

\newenvironment{solution}{\begin{proof}[Solution]}{\end{proof}}
\usepackage{pgfgantt}
\usepackage{graphicx}
\usepackage{xcolor}
\usepackage{verbatim} 


\ganttset{group/.append style={orange},
milestone/.append style={red},
progress label node anchor/.append style={text=red}}
\begin{document}
 
% --------------------------------------------------------------
%                         Start here
% --------------------------------------------------------------
 
\title{Project CATDB}
\author{Bassiro Traore\\ %replace with your name
}

\maketitle

R\'{e}union sur la prise en main du projet CATDB.
Pr\'{e}sents: Bassiro , Jean Phillipe , Veronique
 
% --------------------------------------------------------------
%     You don't have to mess with anything below this line.
% --------------------------------------------------------------

\section{Todo}
Faire des modules pour afficher les éléments

\begin{itemize}
\item faire un tableau \\
Project Title organism organ/tissus technology , experiment name, experiment type number of samples /number of comparaison \\



\item pour le menu prevoir  les onglets  Donwload
                    \\
					Info \\
				      Browse \\
					projet \\
				      oragnism (ensemble d'esp\`{e}ces) \\
				      \\

\item Recherche par mot clé  \\

\item Proposer un carrousel pour les histogrammes \\
\item Pour les histogrammes options clikables  \\
\item Comparaison vue vs requ\^{e}tes en temps par rapport a la requête  (lieu de sauvegarde chips\_tmp option construction de vue) \\

\item choix du la biblioth\`{e}ques (D3js) base sur le tableau des forces et faibles des biblioth\`{e}ques graphiques 
\end{itemize}

\section{Divers}
prochaine r\'{e}union les jeudi 21 11 après le lab meeting
pr\'{e}voir une pr\'{e}sentation avec les items suivant ( But du stage et pr\'{e}sentation du stage)

\section{R\'{e}marque importante}
les colonnes multiples (answers) exemples oragn ont des valeurs s\'{e}pares par des virgules

\section{Requ\^{e}te utile}

\begin{verbatim}
select distinct experiment_type from chips.experiment;
\end{verbatim}


\newpage

\begin{verbatim}
finir le gant sur la base des tâches
\end{verbatim}

     \begin{ganttchart}[%Specs
     y unit title=0.5cm,
     y unit chart=0.7cm,
     vgrid,hgrid,
     title height=1,
%     title/.style={fill=none},
     title label font=\bfseries\footnotesize,
     bar/.style={fill=blue},
     bar height=0.7,
%   progress label text={},
     group right shift=0,
     group top shift=0.7,
     group height=.3,
     group peaks width={0.2},
     inline]{1}{24}
    %labels
    \gantttitle{Semaine 2 18-20 fevrier}{24}\\  % title 1
    \gantttitle[]{2019}{12}  \\                % title 2
    \gantttitle{J1}{3}                      % title 3
    \gantttitle{J2}{3}
    \gantttitle{J3}{3}
    \gantttitle{J4}{3}\\
    % Setting group if any
    \ganttgroup[inline=false]{Tableau}{1}{5}\\ 
    \ganttbar[progress=10,inline=false]{Planning}{1}{4}\\
    \ganttmilestone[inline=false]{Milestone 1}{9} \\

    \ganttgroup[inline=false]{Histograme}{5}{12} \\ 
    \ganttbar[progress=2,inline=false]{Data}{10}{19} \\
    \ganttmilestone[inline=false]{preparation images}{17} \\
    \ganttbar[progress=5,inline=false]{Affichage}{11}{20} \\
    \ganttmilestone[inline=false]{Milestone 3}{22} \\       

    \ganttgroup[inline=false]{Group 3}{13}{24} \\ 
    \ganttbar[progress=90,inline=false]{Task A}{13}{15} \\ 
    \ganttbar[progress=50,inline=false, bar progress label node/.append style={below left= 10pt and 7pt}]{Task B}{13}{24} \\ \\
    \ganttbar[progress=30,inline=false]{Task C}{15}{16}\\ 
    \ganttbar[progress=70,inline=false]{Task D}{18}{20} \\ 
\end{ganttchart}
http://jacob.ips2.u-psud.fr/~jtamby.loc/htdocs/Cahiers/Tamby/documents/CATdb/catdb_mcd.jpg
 
\end{document}

