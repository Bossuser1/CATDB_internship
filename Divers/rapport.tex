
\begin{developpement}

la production de données de visualisation de proccess biologique est un domaine particulier car fait appel au differents état de comparaison de resultats et des facteurs minumamun pour obtenir le même resultat.

la representation de ses données sont d'autant plus importantes que 


\begin{contexte biologique}
repondre a la prise en compte de données de RNA seq dans le processus de visulation des données de CATDB et au processus car ,
CATDB .\\
Et ceux des données collectés par le même processs .\\
\end{}
\begin{Decription des processus}
les processus energistrées dans la base CATDB vont de collecté d'echantillons primaires à la collecte de resultats finaux de traitements aussi bien staistiques , que chimiques en vue de rendre ces resultats reproductibles pour tous utilisateurs.
\end{}

\begin{les differentes types de bases}
Ainsi les Banques et bases de données en Biologie specialisées dans le domaine sont nombreuses , mais on peux compte un certains nombreux d'entre elles specialisées par grands domaines tels que celle specialisées dans :

\ennumerate{}
\item{les ressources pour les procaryotes}

\item{les ressources pour les plantes}

\item{les ressources pour les animaux}

\end{} 

\end{} 
\end{}



#!/usr/bin/env python
# -*- coding: utf-8 -*-

#lecture de requete

def lecture_query_tage(file_query):
    """
    function sert a lire les requetes prédefinir seulement
    """
    query_tag=dict()
    with open(file_query,"r") as fiche:
        data=fiche.readlines()
        for line in data:
            if line[0]!="#":
                if line[0:9]=="Tag_query":
                    _key=line[11:].replace('\n','').replace('"','')
                elif line[0:5]=="query":
                    query_tag[_key]=(line[6:].replace('\n','').replace('"',''))            
    return query_tag
lecture_query_tage("query_tag.ini")
